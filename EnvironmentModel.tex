\subsection{Environment model}

This section elaborates our combined map which represents spatial and semantic knowledge of the environment.

\begin{figure}[htbp] %  figure placement: here, top, bottom, or page
   \centering
   \includegraphics[width=15cm]{images/semanticMap1.pdf} 
   \caption{Block diagram of our \textit{Environment model}}
   \label{Figure: Block diagram of our Environment model}
\end{figure}

Our model represents metric, topological and semantic features of the environment in two separate parts \cite{2}:  S-Box (spatial box) and T-Box (terminological box).
This representation methodology is adopted from the approaches \cite{2, 3, 4, 5, 6} which are mentioned in detail in section~\ref{sec:semantics} .\\

The S-Box represents spatial and symbolic information. 
Spatial information of the entities(\textit{regions} and \textit{objects}) in the environment is stored at the lower level. 
These entities are assigned symbols(labels) which form the top level of S-Box. 
T-Box stores the meaning(semantics) of these entities in the form of concepts and their relations.
In this work, we consider two concepts to represent the environment information: \textit{Regions} and \textit{Objects}.
Symbols in S-Box are instances of these concepts in T-Box.
For instance, considering the home environment setup, the lower level of S-Box may store the metric(occupancy grid) map of a kitchen. This map is assigned a symbol, say, \textit{kitchen-1} and stored at the symbolic (top) level of S-Box.
T-Box would then store the semantic information that kitchen is a type of \textit{Region} and implies a space with some properties and contains objects like fridge, oven, stove etc.

\subsubsection{S-Box}
\begin{figure}[htbp] %  figure placement: here, top, bottom, or page
   \centering
   \includegraphics[width=14cm]{images/sboxexample.pdf} 
   \caption[S-Box of our \textit{Environment Model}]
   {S-Box of our \textit{Environment Model}. Here only connections for room1 and object1 with other levels are shown. 
   Same connections exist for all rooms and objects.
   Bold line indicates separation between two levels of information. Dotted line indicates the layers are at the same level}
   \label{fig:Our Semantic map}
\end{figure}

Information in S-Box is stored in two levels: Spatial level and Symbolic level.\\
This forms a spatio-symbolic hierarchy \cite{3} since the symbolic level is formed over the spatial level.\\\\
Spatial level represents spatial information in four different layers\cite{2, 7}:
\begin{itemize}
 \item \textit{Appearance layer}
 \item \textit{Occupancy layer}
 \item \textit{Navigation layer}
 \item \textit{Topological layer}
\end{itemize}
However, these layers do not form a hierarchical representation. In fact, they are arranged in series 
forming a flat representation. This representation is adopted in order to keep the layers (of spatial level) independent 
of each other, so that if a domain does not require any of these layers, it could be omitted easily without affecting the system. \\

At the symbolic level, instances (symbols) of entities are maintained in one single layer. We term this layer as \textit{Symbolic layer}.  
The \textit{Appearance, Metric, Navigation} and \textit{Topological} layers of the spatial level are each individually linked to the \textit{Symbolic layer} as shown in Figure~\ref{fig:Our Semantic map}. 
For instance, in the figure, the local grid map-1 of the doorway is connected to its instance \textit{doorway-1} at the symbolic level. 
Similarly, the semantic position of this doorway at the navigation level and its node(region-1) at the topological level are connected to its instance \textit{doorway-1}.
Also, as indicated in the figure by dotted bi-directional arrows, this linking provides inferencing in both the directions \footnote[4]{Inferencing in both directions: If our model is queried for instances of kitchen, it will give \textit{kitchen-1} as the answer. 
Also, if the model is queried to which class the symbol \textit{kitchen-1} belongs to, it will give the answer as kitchen}.

\paragraph{Appearance layer}
The \textit{Appearance layer} stores information about the objects in the environment.
It could be object images or pound cloud data obtained from sensors.
However, the focus of this work is on representing information about spatial entities and using the same for navigation.
Hence, we do not go into the details of representing objects in the environment. 
The layer could be considered to be non-existing for this work and could be substituted by any appropriate representation, 
depending on the domain requirement.

\paragraph{{Metric layer}}
The \textit{Metric layer} stores spatial information in the form of local grid-maps with respect to local coordinate system.
These local grid-maps are a representation of the regions existing in the environment.
Longer sides of the regions face towards north in their local coordinate system \cite{1} as shown in the Figure~\ref{fig:cs} below:\\ 
\begin{figure}[htbp] %  figure placement: here, top, bottom, or page
   \centering
   %\includegraphics[angle=90,width=10cm]{images/tbox.jpg}
   \includegraphics[width=9cm]{images/metriclayer.pdf}
   \caption{Definition of local coordinate system for regions based on \cite{1}} 
   \label{fig:cs}
\end{figure}\\
This layer also maintains the shape of the regions representing the geometric area covered by the regions with respect to a global coordinate frame.
The shape of the regions is represented using the method used in \cite{1} where a center rectangle is used along with two connected sub-rectangles as shown in Figure~\ref{fig:geo} below. 
\begin{figure}[htbp] %  figure placement: here, top, bottom, or page
   \centering
   %\includegraphics[angle=90,width=10cm]{images/tbox.jpg}
   \includegraphics[width=9cm]{images/geometry.png}
   \caption[Region geometry represented by triple \cite{8}]
   {Region geometry represented in \cite{8} by triple \textit{(x, y, w, h, $\theta$, $w_{l}$, $h_{l}$, $y_{l}$, $w_{r}$, $h_{r}$, $y_{r}$)}} 
   \label{fig:geo}
\end{figure}

\paragraph{{Navigation layer}}
This layer basically maintains navigable points of a region.
Each region has a navigation point which we call as \textit{semantic position} associated with it \cite{8}.
Regions can have multiple semantic positions as well. 
Even stationary objects like fridge or doors can have \textit{semantic positions}. Robots usually have to manipulate objects like doors,
or they need to go close to 
objects like table for grasping objects like coffee cups kept on the table . 
Semantic positions associated with such objects(doors, tables) would help to serve the purpose of manipulating and/or grasping better.
Even movable objects(like cups) can have semantic positions which would help in grasp planning as mentioned in \cite{18}. 
Our work considers only semantic positions associated with regions.\\

As shown in the Figure~\ref{fig:Our Semantic map} , the semantic positions are encircled indicating their belonging to a particular region. For instance, for \textit{livingroom-1} two semantic positions are encircled: one for the room itself to help the robot to navigate to the living room 
and one for \textit{table-1} to help the robot to grasp some other object placed on the table. The details of semantic positions is elaborated in section ~\ref{sec:semanticposition} .

\paragraph{{Topological layer}}
The \textit{Topological layer} represents topological connectivity between \textit{Regions}.
\textit{Regions} are represented by nodes, like in Figure~\ref{fig:Our Semantic map} , doorway-1 is represented by node \textit{region1}. 
Edges connecting the nodes represent the relation \textbf{neighbourOf} which indicates that \textit{Regions} are topologically connected to their neighbors \cite{1}.
Also, the relative orientation of neighbouring regions is represented in the layer using the concepts: \textbf{northOf, eastOf, southOf} and \textbf{westOf} \cite{1}. The relationships are shown in Figure~\ref{fig:r}.  
\begin{figure}[htbp] %  figure placement: here, top, bottom, or page
   \centering
   %\includegraphics[angle=90,width=10cm]{images/tbox.jpg}
   \includegraphics[width=8cm]{images/topologicallayer.pdf}
   \caption[Topological relations between regions based on \cite{1}]
   {Topological relations based on \cite{1}: C represents \textit{Region} concepts and R their relations}.
   %\textit{Note: Italic text indicates concepts and bold text indicates their relations}} 
   \label{fig:r}
\end{figure}
Maintaining such relationships makes human-robot communication more effective by helping humans in assigning tasks to robot or retrieving information about the robot status in large environments \cite{8}.
For instance, robots could be given a task \textit{``Bring `x' book from the room which is to the north of room `y' "}.
Or the robot can inform the operator \textit{``I am stuck in the corridor which is to the east of room `z' "}.\\

The topological layer  could be viewed as a layer dividing the semantic positions in the navigation layer into regions,
for instance, in the Figure~\ref{fig:Our Semantic map} , region2 represents a region consisting of two semantic positions.
For navigation planning, the topological connectivity between regions helps the planner to generate a symbolic plan which consists of regions the robot
has to navigate through to reach the goal. Based on this symbolic plan, the semantic navigation planner further queries the environment model for the corresponding semantic positions
For instance, if the robot is in the \textit{livingroom-1} and has go to \textit{kitchen-1}, the planner first queries the topological layer about the connectivity between these two regions ({\textit{livingroom-1 $\longleftrightarrow$ doorway-2 $\longleftrightarrow$ kitchen-1}})
and then based on this information, the planner queries the navigation layer for semantic positions of these regions.

\paragraph{{Symbolic layer}}
At this layer, entities in the environment are represented in the form of symbols(labels).
These symbols are instances of concepts(\textit{Region} and \textit{Object}) in T-Box.
%Maintains environment information in the form of instances of entities (\textit{Region} and \textit{Object}).
%An instance of a new entity is created by asserting a unique symbol(label) to it and linking this symbol to the corresponding concept (in T-box).
Whenever an instance of a new entity is created it is linked to its corresponding concept in T-box by assigning a unique symbol to it and asserting this symbol to the corresponding concept in T-box.
%Lower layers basically maintain metric, topological and semantic properties of these instances.
For instance, suppose a new local grid map of a region, say, living room is created. This map is be asserted a symbol \textit{livingroom-1} and
is linked to the concept \textit{Region} (or \textit{LivingRoom} to be precise, which is a sub-category of \textit{Region}) in the T-Box. 
Similarly, the semantic positions and topological relations are asserted to the symbol of the new entity created. 
This process of assertion and linking could be done by hand(by a human operator) while creating the environment representation using some knowledge representation language \footnote[5]{In \cite{2} LOOM knowledge representation system\cite{24} is used to create the concepts and relations of the T-Box.
In \cite{3, 4} NeoClassic system\cite{25} is used to create concepts and the linking between the sensor data(grid map) and symbols is done using a process called Anchoring \cite{23}. \cite{22, 17} uses Web Ontology Language(OWL)\cite{26} while \cite{1} uses ObjectLogic\cite{27}}.
It could also be taken care of by an on-line SLAM algorithm, like the one proposed in \cite{8} which is capable of creating an environment representation with semantic, metric and topological features.\\

Although the focus of this work is not on representing object entities, for now,we group symbols of objects based on their geometric positions and 
use human-like environment distribution(that is, objects \textit{are located} in rooms)\cite{3}.
For instance, in Figure~\ref{fig:Our Semantic map} , objects(\textit{tvset-1, plant-1} etc) belonging to the living room are grouped together and linked to \textit{livingroom-1} symbol via \textbf{at} relation.
However, different criteria could be used depending on the application. For instance, objects could be grouped based on their mobility \cite{21, 22}.\\
\textit{Example:} Structural objects(walls,floors) could be grouped together, while furniture objects
(table, chair etc) will form another group.


\subsubsection{T-Box}
\begin{figure}[htbp] %  figure placement: here, top, bottom, or page
   \centering
   %\includegraphics[angle=90,width=10cm]{images/tbox.jpg}
   \includegraphics[width=10cm]{images/tbox3.pdf}
   \caption{\textit{Region} and \textit{Object} concept and their relation}
   \label{fig:T-Box}
\end{figure}
\begin{figure}[htbp] %  figure placement: here, top, bottom, or page
   \centering
   %\includegraphics[angle=90,width=10cm]{images/tbox.jpg}
   \includegraphics[width=15cm]{images/tbox1.pdf}
   \caption{Extension of \textit{Region} concept}
   \label{fig:TBox2}
\end{figure}
\begin{figure}[htbp] %  figure placement: here, top, bottom, or page
   \centering
   %\includegraphics[angle=90,width=10cm]{images/tbox.jpg}
   \includegraphics[width=12cm]{images/tbox2.pdf}
   \caption{Extension of \textit{Object} concept}
   \label{fig:TBox1}
\end{figure}
The T-Box stores semantic knowledge about the robot environment. 
We refer to this knowledge as semantic because it assigns meaning to the information represented in the S-Box.
The knowledge includes the types of entities(regions and objects) found in the environment along with their relationships.
It is represented in the form of concepts and relations which are structured in hierarchical form called ontology \cite{2}.
Our ontology consists of two concepts: \textit{Region} and \textit{Object} which are related by \textbf{located-at} relation 
as shown in Figure~\ref{fig:T-Box} .
Further sub categorization of \textit{Region} and \textit{Object} concepts is done until reaching most specific categories like 
\textit{kitchen, living room, office room} etc.
%For this work the focus is on representing spatial properties of regions in the environment to help plan navigation. So the \textit{Object} concept is not elaborated in detail in the ontology. 
% (But the representation will be flexible to incorporate as many Object categories and properties  as required by the application domain) 
Since the focus of this work is on navigation we consider detailed categorization of \textit{Region} which is shown in Figure~\ref{fig:TBox2} ,
considering two different environment setups (Home and University) mentioned in section~\ref{sec:har}.
Also, general categorization of \textit{Object} concept is shown in Figure~\ref{fig:TBox1} .\\

Along with representing the entities and their relations in hierarchical form(ontology), the T-Box also stores general knowledge about them by
assigning meaning to concepts(entities) in the ontology.
For instance, the T-Box stores the information that \textit{Kitchen} is a \textit{Room} containing a \textit{Fridge} and \textit{Stove}. 
This is depicted in the Figure~\ref{fig:SM} :

\begin{figure}[htbp] %  figure placement: here, top, bottom, or page
   \centering
   %\includegraphics[angle=90,width=10cm]{images/tbox.jpg}
   \includegraphics[width=4cm]{images/tbox4.pdf}
   \caption{Definition of \textit{Kitchen} concept in T-Box}
   \label{fig:SM}
\end{figure}

\paragraph{Our T-Box ontology}
As mentioned above, the focus is on the concept \textit{Region}.
The \textit{Region} concept is categorized into: \textit{Room, Passage, Corridor} and \textit{Junction}.
\begin{itemize}
 \item \textit{Room:} This concept is further sub-categorized into concepts like \textit{Kitchen}, \textit{Living room}, \textit{Office room} and other types 
 of rooms found in human environment.
 \item \textit{Passage:} This concept represents regions which help the robot to transit from one region to other. The concept \textit{Passage} is further categorized into:
 \begin{itemize}
  \item \textit{Horizontal Passage} which includes \textit{Doorway} (and \textit{Window}).
  \item \textit{Vertical Passage} is further categorized as \textit{Elevator}.
 \end{itemize}
The reason for treating \textit{Doorway}, \textit{Elevator} different from \textit{Room} is to help navigation planner adopt different strategies (if required) 
while navigating through narrow doorways, elevators. 
This also helps in reasoning and recovering in case of failure while navigating through the doorways, elevators . 
 \item \textit{Corridor:} This concept represents long regions connecting different rooms.
 \item \textit{Junction:} In this work we introduce a new concept called \textit{Junction} to deal with situations, where two or more corridors intersect. 
Such a region is called a \textit{Junction} in our approach.
Figure~\ref{fig:jn} shows an example:
\begin{figure}[htbp] %  figure placement: here, top, bottom, or page
   \centering
   %\includegraphics[angle=90,width=10cm]{images/tbox.jpg}
   \includegraphics[width=6cm]{images/tbox5.pdf}
   \caption{\textit{Junction} concept example}
   \label{fig:jn}
\end{figure}
\end{itemize}


\iffalse 
\subsubsection{ Semantic SLAM:}
\begin{itemize}
 \item This work will make use of SLAM approach similar to the one developed in \cite{8}.
 \item The approach uses FastSLAM 2.0 capturing region features with semantic, topological as well as metric properties.
 \item It assigns semantic meaning to the regions (e. g., rooms, offices,hallways, doors, desks, cupboards, etc.), and also assign topological relation between these regions as well as maintain 
 geometric layout of each region.
 \item \textit{Semantic description:} Includes the type of the sub-category of the region class to which the region under consideration belongs to. To deal with uncertainity, a confidence value is assigned.
 \item In this work, we include additional information at the semantic level. This includes adding semantic positions to each instance of region class. Figure 6 depicts the semantic description part of the data structure with addition of semantic position. 
 \item \textit{Topological description:} Connectivity to the neighboring regions and also hierarchical links to sub-regions. The connectivity is associated with a confidence value to deal with uncertainity.
 \item \textit{Geometric description:} Regions' geometric is modeled by three rectangles along horizontal axis. The entire shape could be rotated arbitrarily. Shown in figure 7.

 \item Figure below shows the representation used in \cite{8} to store the region features.
  \begin{figure}[htbp] %  figure placement: here, top, bottom, or page
    \centering
    %\includegraphics[angle=90,width=10cm]{images/tbox.jpg}
    \includegraphics[width=10cm]{images/SLAM.png}
    \caption{Storing semantic, topological and metric features \cite{8}}
    \label{Fig:} 
  \end{figure} 
  \begin{figure}[htbp] %  figure placement: here, top, bottom, or page
    \centering
    %\includegraphics[angle=90,width=10cm]{images/tbox.jpg}
    \includegraphics[width=10cm]{images/semanticposition.png}
      \caption{Semantic description part after addition of semantic position details}
    \label{Fig: Semantic description part after addition of semantic position details}
  \end{figure} 
\end{itemize}
\begin{figure}[htbp] %  figure placement: here, top, bottom, or page
   \centering
   %\includegraphics[angle=90,width=10cm]{images/tbox.jpg}
   \includegraphics[width=10cm]{images/geometry.png}
   \caption{}
   \label{Fig: Method of storing geometric shape of the room}
\end{figure}
\fi