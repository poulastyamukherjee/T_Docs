\section{Introduction}
\subsection{Motivation}
Robotic Welding constitute majority of all robotic applications in industries today [1]. Large scale
industries with fixed manufacturing processes - like the car industry - traditionally uses MIG/MAG
welding operations in the car body workshops of the assembly lines. Due to falling costs of
robots and increasing precision and greater computational prowess, there is an increasing
number of smaller businesses, which follow client oriented manufacturing and produce unique
products designed specifically for each client- can now afford to use robotic manufacturing
processes to deliver highly customizable and precise goods.
Welding being one of the most common industrial process has been a subject of lot of ongoing
research for finding out the most suitable way to automate the task by using robots[1]. That
being said robotic welding is not at all a straightforward task because, it is difficult to
parameterize and to ensure weld quality assessment and control[1]. However robotic welding to
significant improvement both in terms of improvement in quality of product and reducing
manufacturing costs.
Along with the significant challenges mentioned above, robotic welding requires significant
knowledge both in the domain of the welding process - which skilled workers usually acquire
with years of experience - and programming the robot to perform the a welding task even on
simple workpieces[2]. A solution prevalent in the industry is the use of a teach pendant to guide
the robot’s manipulator for each task, in our case a weld seam, and then let it perform the task
autonomously[1,2,3]. The main disadvantage of this method is the significant longer teaching
time involved with each new workpiece. Thus the need of the hour is a simple programming
interface which will include the process knowledge of welding[1] and allow an operator to
generate programs for different weld seams without having to reprogram the robot. Even
though generating a welding program is quite simple, the complexity of the process increases
when several constraints are introduced - maintaining constant velocity during welding,
maintaining constant angle between tool tip and weld seam. The welding process can also be
affected by collision objects and presence of singular points in the path of the weld tool tip. While
avoiding collision is taken care of in the planning level, modern controllers are programmed to
handle singularities. However this kind of singularity avoidance usually results in deviation of the
behavior of the robot, which can result in poor quality weld surface.
An architecture- namely PPR (Product, Process and Robot) - has been proposed in [8] which uses
separate models for workpiece, robot and welding process and generates motion plans while
taking into account the process constraints. Due to the fact it uses sample based planning
algorithms, it can easily generate programs in new workspace environments as well as for new
3
workpieces with varying levels of complexity (linear edges, circular edges, discontinuous edges
etc).
\subsection{Overview}
\subsubsection{Objective}
The main objective of this work is to extend the existing RobotKit[8] software with multi-robot
planning capability. A 6 Dof manipulator will be the primary robot while a 2 dof rotary table will fulfill the role of second robot. Quite a number of multi robot approaches involving a
manipulator and a rotary table and two manipulators - already exists. Most notable among them
are Master/Slave[5] and non-Master/Slave approach[6], centralized[5] and decentralized
planning approach[6]. Our work will also evaluate the most suitable method among these based
on criterias like performance time and motion cost as mentioned in[1].
The second objective of the work is to improve the time of solution generated and reduce
solution cost, or in other words ensure greater compliance with existing manufacturing process
model[8].
The third objective of the work is to extend the existing State Validation model to include
Singularity and Reachability checking to generate more robust solution path.
\subsubsection{Proposed Contributions}
\begin{enumerate}
\item Extending the robot model of the PPR architecture from single robot welding process to
multiple robot welding approach .
\item Extending the state validation model of the PPR architecture to include singularity and
reachability check to ensure greater compliance with process model.
\item Improve performance of the planning algorithm by incorporating cost information and
replacing the randomized sampler with a sampler that samples based on the cost
information.
\item Evaluation of the proposed methods based on the performance changes in time of
generating solution and final solution cost.
\item Implementing other types of planning algorithms e.g. Multi query planners such as PRM
(Probabilistic Roadmap), SPArse Roadmap Spanner algorithm (SPARS), and other Single
query planners such as Rapidly-exploring Random Trees (RRT), Expansive Space Trees
(EST) and evaluating the performance with respect to the current problem in hand and
determining the best suited planning algorithm.
\item Evaluation of performance of the proposed changes of this work compared to existing
system based on time required to generate an optimal solution and cost of final solution.
\end{enumerate}
\subsection{Structure}
\begin{itemize}
 \item Chapter 2 describes the state of the art work in the fields of robotic path planning, multi robot path planning, robotic welding and optimizing weld planning.
 \item Section 3 describes the formulated problem and the plausible approach to solve it.
 \item Section 4 describes the approach taken to solve the initial position optimization of the table.
 \item Section 5 describes the experimental setup and the results of the experiment.
 \item Section 6 describes a use case in which the algorithm was implemented and subsequent results visualized.
 \end{itemize}






