\section{Conclusion}
\subsection{Contributions}
In this work we proposed a geometry based approach to area coverage problem using UAVs. We proposed that since UAVs have limited computational capabilities, using algorithms which require very little computational power can be helpful in creating better solutions. Based on the experiments carried out we came to the conclusion that for doing coverage of small areas lawnmower approach is more suited as it ensures complete coverage in minimal time, however from table ~\ref{tab:exc5qw6256}, we can clearly conclude that spiral method is the best method overall as it ensures atleast 98 \% coverage while having minimum energy cost and time to completion. Along with the spiral method, the hybrid method also shows some promise, but its performance in small areas is really poor. The RRT method does not guarantee complete coverage. \\
Regarding the fact that a smooth motion of the UAVs during its course of flight is a requirement not only for capturing information uniformly but also ensuring a high level of flight stability, the results published in ~\ref{subsubsec:smoothy3} clearly show that the spiral method gives the best performance as it ensures its acceleration and deceleration are within a certain range thus ensuring greater flight stability

\subsection{Possible Improvements and Future Directions}
A very important future work is to implement the proposed algorithms on a real system and determine whether their performance is in accordance with the findings of this work. In terms of improving the performance of the algorithm, introducing machine learning to make the UAVs autonomously decide which algorithm to use for which scenario will be a big improvement. The primary requirement for creating such a system would be to run multiple number of test to gather enough data. \\

Apart from the above mentioned directions, changing the orientation and position of the sensor on the UAV and studying how it affect coverage performance is also a possible extension of this work.
