\subsection{Case Study}
\subsubsection{Home environment for personal assistant robot}
\begin{figure}[htbp] %  figure placement: here, top, bottom, or page
   \centering
   %\includegraphics[angle=90,width=10cm]{images/tbox.jpg}
   \includegraphics[width=12cm]{images/example.pdf}
   \caption{Home environment setup}
   \label{Test environment map}
\end{figure}
In this section with the help of an example we explain the working of our proposed system.
Considering the home environment setup mentioned in section~\ref{sec:har} the robot is given the goal of ``go to kitchen".
The current semantic location of the robot is taken to be \textit{livingroom-1}.
Thus the robot has to start from \textit{livingroom-1} and go to \textit{kitchen-1}.\\

Our proposed \textit{Navigation Planner System} works as elaborated below:\\ 
First, the \textit{Semantic Navigation Planner} unit receives the goal ``go to \textit{kitchen-1}" from the high level (task/symbolic) planner.
Then it queries the \textit{Environment Model} about the topological connectivity between the two \textit{Region} instances.
In this example the two regions are connected as follows:
\textit{livingroom-1} $\longleftrightarrow$ \textit{doorway-2} $\longleftrightarrow$ \textit{kitchen-1}\\
The \textit{Semantic Navigation Planner} unit then queries the \textit{Environment Model} for corresponding semantic positions for each of the above region instances.
Thus, the symbolic goal of ``go to the kitchen" is converted to geometric sub-goals: \textit{go to SP2}, \textit{go to SP3}, where \textit{SP2} represents semantic position for Doorway-2 and \textit{SP3} for Kitchen-1.
These sub-goals along with associated constraints are stored in the \textit{Scheduler} component of the \textit{Semantic Navigation Planner} unit .
It is assumed that \textit{SP2} has a medium tolerance and \textit{SP3} has a soft tolerance and the other semantic properties and constraints are neglected.
Each of these sub goals is then fed to the \textit{Monitoring} component which further passes it to the motion planner.
Once the sub goal \textit{SP3} is achieved, the motion planner triggers an event, notifying the completion of the sub goal to the \textit{Monitoring} component.
After receiving this notification the \textit{Monitoring} component feeds the next sub goal and this continues till the last sub goal is achieved.\\

\textbf{Dealing with unexpected situations:}\\
\textit{Situation 1:}\\
Suppose the robot is not able to reach \textit{SP2}.
The motion(path) planner notifies the \textit{Monitoring} component about this failure by triggering a failure event.
The \textit{Monitoring} component then notifies the \textit{Navigation Failure Analyzer} about this goal/sub goal failure.
To identify the cause of failure the \textit{Navigation Failure Analyzer} queries the \textit{Environment Model} for current environment and robot's state.
We assume \textit{doorway-2} has a automatic \textit{door-2} since in this work we consider doorways with only automatic doors.
Based on this information, the \textit{Failure Analyzer}  concludes the cause of failure as malfunctioning of \textit{door-2} at \textit{doorway-2} (assuming the \textit{Failure Analyzer} is capable of detecting such failures).
Since in such situations it is not possible to navigate through \textit{doorway-2}, the \textit{Navigation Failure Analyzer}
notifies the high level planner about the goal failure.\\

\textit{Situation 2:}\\
To demonstrate the flexibility provided by the tolerance limit we consider the situation where the goal \textit{SP2} is achieved and the robot is now approaching \textit{SP3}.
Since \textit{SP3} has \textit{soft} tolerance limit, the planner concludes the sub goal to be achieved as soon as the robot is in the kitchen.
This proves to be helpful in situations where there is a dynamic obstacle on or in the close vicinity of the semantic position.

%\subsubsection{Case 2: University environment for library assistant robot}
