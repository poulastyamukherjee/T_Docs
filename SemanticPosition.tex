\subsection{Semantic positions}
\label{sec:semanticposition}
\subsubsection{Representing Semantic positions}
This work represents semantic positions associated with regions by three properties namely:
  \begin{itemize}
    \item \textit{Location}
    \item \textit{Orientation} 
    \item \textit{Tolerance} 
  \end{itemize}
\textit{Location} and \textit{Orientation} concepts are adopted from the approach in \cite{1}, while the \textit{Tolerance} concept is the contribution of this work.
The \textit{Location} and \textit{Orientation} properties are concerned with the x-y co-ordinate of the robot and its rotation angle with respect to local reference frame of that region.
The \textit{Tolerance} property determines how close the robot is expected to go to the semantic position.\\

These properties are represented by symbols and not by numerical values \cite{1}.
The instances of symbols for \textit{Location, Orientation} and \textit{Tolerance} properties used in this work are mentioned below:
\begin{itemize}
  \item \textit{Location} of SP: center/east/west/north/south/north-east/north-west/south-east/south-west
  \item \textit{Orientation} of SP: east/west/north/south
  \item \textit{Tolerance} for SP: soft/medium/hard
\end{itemize}
An example of a semantic position for \textit{kitchen-1} instance of our home environment setup is shown below:
\begin{itemize}
  \item \textit{Location}: center
  \item \textit{Orientation}: north
  \item \textit{Tolerance}: soft/medium/hard
\end{itemize}
The definitions/equations of these symbols are stored in the \textit{Navigation Equations component} of the architecture elaborated in section~\ref{sec:ne} .
However, depending on the application domain, the symbols could be added/removed/modified along with their definitions in the \textit{Navigation Equations component}.\\

Usually rooms and other categories of \textit{Region} concept will have soft or medium tolerance.
\textit{Objects} will mostly have hard tolerance.
These assumptions may change depending on the application domain.

\subsubsection{Advantage of \textit{Tolerance} limit} 
\textit{Tolerance} limit for a semantic position makes navigation flexible in the following way:
Usually a robot goes to a room and then proceeds towards some object.
For instance, in our home environment setup, suppose the robot is given the task of ``bringing a milk bottle". The robot needs to go to the
kitchen and dock a fridge to pick-up the milk bottle. In such scenarios, the robot does not need to reach the exact semantic position of the kitchen.
%it is acceptable if the robot does not exactly reach the kitchen's semantic 
%position since there its is not expected to manipulate any object. Instead it has to approach the fridge after entering the kitchen. 
Assigning soft tolerance to the semantic position of the kitchen enables the robot to conclude the accomplishment
of the goal of ``going to kitchen"  as soon as it is in the kitchen.\\

For the same task, after reaching the kitchen the robot has to dock the fridge to grasp a milk bottle. Hence, fridge could have medium tolerance limit indicating
that robot has go near to the fridge. This nearness depends on the workspace of the robot platform. A hard tolerance limit indicates that the robot has to be at the exact position.
Assigning a hard or soft tolerance limit to the semantic position of an object depends on the workspace of the robot. Robots with limited workspace need to
have hard tolerance to grasp objects. However, regions can always have a soft tolerance limit, irrespective of the platform workspace,
unless robots are not grasping any objects at those positions.\\

The tolerance limit helps to deal with situations in which dynamic obstacles in the vicinity or on the semantic positions. 
It also improves the path planning efficiency by reducing the goal completion time.