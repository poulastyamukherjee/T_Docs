\subsection{Overview} 
\begin{figure}[htbp] %  figure placement: here, top, bottom, or page
   \centering
   \includegraphics[width=17cm]{images/Architecture3.pdf} 
   \caption{Component view of the Architecture}
   \label{fig:SysArc}
\end{figure}

Figure~\ref{fig:SysArc} shows the component view of our proposed system architecture. It is similar to the one proposed in \cite{1}.
It is divided into two parts: An \textit{Environment Model} which stores the information about the environment and a \textit{Navigation Planning System} 
which uses the environment information to generate navigation plans.
They are both indicated by grey color blocks in the Figure~\ref{fig:SysArc}. The dotted boxes are assumed to be existing/working in the system. 
The \textit{Navigation Planning System} is further divided into three components: Semantic Navigation Planner, Navigation Knowledge base and Navigation Failure Diagnosis.
The sections below explain both parts of our architecture in detail starting with the \textit{Environment Model} and then the \textit{Semantic Navigation Planner}.\\

We consider the home environment setup of scenario 2 mentioned in section~\ref{sec:har} for explaining the architecture. 
It is assumed that the present location of the robot is \textit{livingroom-1} and it gets the goal of ``go to kitchen" from the symbolic(task) planner.  
