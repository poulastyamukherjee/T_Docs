\begin{abstract}
Motion planning for welding robots, usually focus on generating plans with only the 6 degrees of freedom manipulator. What is often overlooked is the position of the edge to be welded with respect to the manipulator. Welding cells which come equipped with a rotary table, provides some extra degrees of freedom, thus greater flexibility for program generation. In this thesis work, we model the behavior of a 2 degree of freedom rotary table for a welding cell and incorporate it with the 6 degree of freedom manipulator for generating optimal weld path plans. We design a cost function which is then used to establish the relationship between the path traced by the manipulator and the edge to be welded. We develop a simulated annealing algorithm based approach, to find the optimum position of the workpiece for which the defined cost function reaches a minimum value. Since the performance of simulated annealing depends on various parameters like cooling schedule and cooling rate - we carry out further analysis to determine the optimum parameters for our use case which strikes a balance between finding the optimum and speed of convergence. Next we apply this developed approach to solve a very common problem, which occurs when the edge of the workpiece to be welded lies outside the reachable workspace of the robot. Using the approach developed, we not only reposition the workpiece to a reachable position, but also ensure that it is the most optimum as well. Finally we carry out a benchmarking test amongst some well known planners, for both and optimized unoptimized workpiece positioning to demonstrate that our approach improves the overall quality of generated planned paths.
\end{abstract}
